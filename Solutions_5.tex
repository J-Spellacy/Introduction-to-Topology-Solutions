1. Let $f:A \rightarrow B$ be given. Prove the following:
(a) For each subset $X \subset A, X \subset f^{-1}(f(X))$.

Firstly, from the definition of a function: $f:A \rightarrow B, x \in A, f(x) \in B \iff x' = x \implies f(x) = f(x')$  essentially given a value $x$ there is only one unique $f(x)$ but it is not necessarily true that $x$ is unique for $f(x)$ $f(x) \in B, f^{-1}(f(x)) = X, X \subset A, x \in X$

given then that $X \subset A, f(X) =  y \iff \forall x \in X, f(x) = y, x \in f^{-1}(y)$ and that $\forall x, x' \in X, f(x) = f(x')$
$\therefore \forall x \in X, x \in f^{-1}(f(x))$
$\therefore X \subset f^{-1}(f(X))$

(b) For each subset $Y \subset B, Y \supset f(f^{-1}(Y))$.

given $Y \subset B$ and from the earlier definition we know that $f^{-1}(Y) = \{ x \mid x \in A, f(x) \in Y, x \in f^{-1}(f(x))\}$
$\therefore y \in f(f^{-1}(Y)) \iff y = f(x), x \in f^{-1}(Y)$
$\therefore y \in f(f^{-1}(Y)),  f(x) \in Y $
$\therefore y \in Y, f(f^{-1}(Y)) \subset Y$

(c) If $f:A \rightarrow B$ is one-one, then for each subset $X \subset A$, $$f^{-1}(f(X)) = X.$$
from part a, $X \subset A \therefore X \subset f^{-1}(f(X))$, given that $f(x)$ is one-one, $f(x) = y, f^{-1}(y) = x, \forall x, x' \in A, f(x) = f(x'), \forall y, y' \in B, f(y) = f(y')$
essentially imaging $f^{-1}(y) = x$ as a function which takes $y$ and maps to $x$ so behaves the same as the function in part a (only possible since $f:A \rightarrow B$ is one-one).
$\therefore \forall y \in f(X), f^{-1}(y) \in X$
$\therefore X \subset f^{-1}(f(X)) \subset X$
$\therefore f^{-1}(f(X)) = X$

(d) If $f: A \rightarrow B$ is onto, then for each subset $Y \subset B$, $$f(f^{-1}(Y)) = Y.$$
since $f: A \rightarrow B$ is onto, $f(A) = B, \forall y \in  f(A), y \in B$, $\forall y \in B, y \in f(A)$
using the proof in part b we have shown that $\forall y \in f(f^{-1}(Y)), y \in Y$ 
now $\forall y \in Y, y \in f(f^{-1}(Y))$ as per the definition of an onto or surjective function

2. Let $A = \{ a_{1}, a_{2}\}$ and $B = \{ b_{1}, b_{2}\}$ be two sets, each having precisely two distinct elements. Let $f:A \rightarrow B$ be the constant function such that $f(a) = b_{1}$ for each $a \in A$.
(a) Prove that $f^{-1}(f(a_{1})) \not = \{a_{1}\}$. [Thus it is usually the case that $f^{-1}(f(X))$ and $X$ are not equal.]

$f^{-1}(f(a_{1})) = \{ x \mid f(x) = f(a_{1})\} \therefore a_{2} \in f^{-1}(f(a_{1}))$
$a_{2} \not \in \{ a_{1}\} \therefore f^{-1}(f(a_{1})) \not = \{a_{1}\}$

(b) Prove that $f(f^{-1}(B)) \not = B$. [Thus it is usually the case that $f(f^{-1}(B))$ and $B$ are not equal.]

$B = \{ b_{1}, b_{2}\}, f^{-1}(B) = f^{-1}(\{ b_{1}, b_{2}\}) $ the function is always equal  to $b_{1} \therefore f^{-1}(B) = f^{-1}(b_{1})$
and since $\forall a \in f^{-1}(b_{1}), f(a) = b_{1}$ by definition of the inverse of a function and constant functions,
$\therefore f(f^{-1}(B)) = b_{1} \not = B$ only the  case when $B = \{ b_{1}\}$ (confusing if this is true based on the axiom of regularity, I.e. $x \not = \{ x \}$

(c) Prove that $f(\{a_{1}\} \cap \{ a_{2}\} ) \not = f(\{a_{1}\}) \cap f(\{ a_{2}\})$. [Thus it is not usually the case that $f(X \cap X') \not = f(X) \cap f(X')$.]

$\{ a_{1}\} \cap \{a_{2}\} = \emptyset$, $f(\{ a_{1}\} ) = b_{1}, f(\{a_{2}\}) = b_{1} \therefore f(\{a_{1}\}) \cap f(\{ a_{2}\}) = \{b_{1}\} \not =  \emptyset$
$\therefore f(X \cap X') = f(X) \cap f(X') \iff \forall x, x' \in X \cap X', f(x) = f(x')$

4. Let $f: A \rightarrow B$ be given and let $\{ Y_{\alpha}\}_{\alpha \in I}$ be an indexed family of subsets of $B$. Prove:
(a) $f^{-1}(\cup_{\alpha \in I}Y_{\alpha}) = \cup_{\alpha \in I} f^{-1}(Y_{\alpha}) $.

$f^{-1}(\cup_{\alpha \in I} Y_{\alpha}) = \{a \mid a \in A, \exists\alpha \in I\ s.t.\  f(a) \in Y_{\alpha} \}$ 
$a \in \cup_{\alpha \in I} f^{-1}(Y_{\alpha}) \iff a \in A, \exists\alpha \in I\ s.t.\  f(a) \in Y_{\alpha} $
since they hold equivelant conditions for truth, they only exist if they are equal
$\therefore f^{-1}(\cup_{\alpha \in I}Y_{\alpha}) = \cup_{\alpha \in I} f^{-1}(Y_{\alpha})$

(b) $f^{-1}(\cap_{\alpha \in I}Y_{\alpha}) = \cap_{\alpha \in I} f^{-1}(Y_{\alpha}) $.


$f^{-1}(\cap_{\alpha \in I} Y_{\alpha}) = \{a \mid a \in A, \forall\alpha \in I, f(a) \in Y_{\alpha} \}$ 
$a \in \cap_{\alpha \in I} f^{-1}(Y_{\alpha}) \iff a \in A, \forall\alpha \in I, f(a) \in Y_{\alpha} $
since they hold equivelant conditions for truth, they only exist if they are equal
$\therefore f^{-1}(\cap_{\alpha \in I}Y_{\alpha}) = \cap_{\alpha \in I} f^{-1}(Y_{\alpha})$

(c) If $X \subset B$ then $f^{-1}(\complement(X)) = \complement(f^{-1}(X))$.


$f^{-1}(\complement(X)) = \{ a \mid f(a) \in B, f(a) \not \in X\}$
$a \in \complement(f^{-1}(X)) \iff f(a) \not \in X, a \in A\ s.t.\ f:A \rightarrow B$ since $X \subset B, f^{-1}(X) \subset f^{-1}(B)$
$\therefore a \in \complement(f^{-1}(X)) \iff f(a) \not \in X, f(a) \in B$
since they hold equivelant conditions for truth, they only exist if they are equal
$\therefore f^{-1}(\complement(X)) = \complement(f^{-1}(X))$

(d) If $X$ is a subset of $A$, and $Y$ is a subset of $B$, then $f(X \cap f^{-1}(Y)) = f(X) \cap Y$.

$f(X \cap f^{-1}(Y)) = \{ y \mid y \in f(f^{-1}(Y)), y \in f(X)\}$
$\therefore = \{y \mid y \in Y, y \in f(X)\}$ since by definition $Y =  f(f^{-1}(Y))$
$y \in f(X)\cap Y \iff y \in f(X), y \in Y$
since they hold equivelant conditions for truth, they only exist if they are equal
$\therefore f(X \cap f^{-1}(Y)) = f(X) \cap Y$

5. Let $A$ and $B$ be sets. The correspondence that associates with each element $(a, b) \in A \times B$ the element $p_{1}(a, b) = a$ is a function called the \textit{first projection}. The correspondence that associates with each element $(a, b) \in A \times B$ the element $p_{2}(a, b) = b$ is a function called the \textit{second projection}. Prove that if $B \not = \emptyset$, then $p_{1}:A \times B \rightarrow A$ is onto and if $A \not = \emptyset$, then$p_{2}:A \times B  \rightarrow B$ is onto. Under what circumstances is $p_{1}$ or $p_{2}$ one-one? What is $p_{1}^{-1}(\{a\})$ for an element $a \in A$?

Let $B \not = \emptyset \ s.t.\ \exists b \in B $ and $p_{1}:A \times B \rightarrow A$ is not onto,
$\therefore p_{1}(A\times B) \not = A \therefore \exists (a, b) \in A \times B\ s.t.\ p_{1}((a, b)) \not \in A$ or $\exists a \in A\ s.t.\ a\not \in p_{1}(A\times B)$,
$p_{1}(A \times B) = \{ a \mid  a \in A\} \therefore \forall a \in A, a \in p_{1}(A \times B)$,
since if $a \in A, b \in B, (a, b) \in A \times B \therefore a \in p_{1}((a, b)), a \in A$.
Therefore given the conditions $B \not = \emptyset \ s.t.\ \exists b \in B $ it must be true that $p_{1}:A \times B \rightarrow A$ is onto.

The same logic applies with the conditions of $A \not = \emptyset$ and $p_{2}(A \times B) \rightarrow B$.

From this we have established that $p_{1}$ is one-one when $B$ has just one unique element $B = \{ b\}, A \times B = \{ (a_{i}, b)\}_{\forall a_{i} \in A}$ because each instance of the product set is a unique $a$ with the same $b$ and thus can only be given from $p_{1}(A \times B) = \{ a \mid  a \in A\}, \forall (a, b) \in A \times B$ there are no elements in $A \times B$ that repeat the $a$

$p_{1}^{-1}(\{ a\}) = \{(a, b) \mid \forall (a, b), (a', b') \in A \times B, a = a' \}$

6. Let $A$ and $B$ be sets, with $B \not = \emptyset$. For each $b \in B$ the correspondence that associates with each element $a \in A$ the element $j_{b}(a) = (a, b) \in A \times B$ is a function. Prove that for each $b \in B, j_{b}:A \rightarrow A \times B$ is one-one. What is $j_{b}^{-1}(W)$ for a subset $W \subset A \times B$?

For $j_{b}:A \rightarrow A \times B$ for a given $b \in B, B \not = \emptyset$ so that $b' = b, \forall(a, b), (a, b') \in j_{b}(A)$,
Let$j_{b}:A \rightarrow A \times B$ not be one-one, since $j_{b}(a) = (a, b), j_{b}(a') = (a', b)$ $\therefore \exists a , a' \in A\ s.t.\ (a, b) = (a', b), a \not = a'$ 
by definition of an ordered pair, if $a \not = a', (a, b) \not = (a', b)$
thus given $b \in B$ (so for each $b \in B$) $j_{b}:A \rightarrow A \times B$ is one-one
if we vary $b$ the same is true but over both components of an ordered pair; $j_{b}(a) = (a, b), j_{b'}(a') = (a', b')$, $(a, b) = (a', b') \iff b = b', a = a'$

$j_{b}^{-1}(W) = \{ a \mid (a, b) \in W\}$ since the function can be run $\forall (a, b) \in W, \not \exists (a, b) \in W\ s.t.\ (a,b) \in j_{b}^{-1}(W)$ because $j_{b}$ is one-one

7. Let $A$ be a set and $E \subset A$. The function $\mathscr{x}_{E}: A \rightarrow \{ 0, 1\}$ defined by $\mathscr{x}_{E}(x) = 1$ if $x \in E$ and $\mathscr{x}_{E}(x) = 0$ if $x \not \in E$ is called the \textit{characteristic function} of $E$. Let $E$ and $F$ be subsets of $A$, show:

(a) $\mathscr{x}_{E \cap F} = \mathscr{x}_{E}\cdot \mathscr{x}_{F}$, where $\mathscr{x}_{E} \cdot \mathscr{x}_{F}(x) = \mathscr{x}_{E}(x) \mathscr{x}_F(x)$. 

$\mathscr{x}_{E\cap F}(x) = 1 \iff  x \in E, x \in F$
$\mathscr{x}_{E}(x) = 1 \iff x \in E$, $\mathscr{x}_{F}(x) = 1 \iff x \in F$
$\therefore \mathscr{x}_{E}(x) \mathscr{x}_F(x) = 1 \iff x \in E, x \in F$
alternatively they all equal $0$ in either case for the same $x$ $\mathscr{x}_{E \cap F} = \mathscr{x}_{E}\cdot \mathscr{x}_{F}$

(b) $\mathscr{x}_{E \cup F} = \mathscr{x}_{E} + \mathscr{x}_{F} - \mathscr{x}_{E \cap F}$ and find a similar expression for $\mathscr{x}_{E \cup F \cup G}$.

let $B \subset A$, 
$\forall x \in B$ if $x \in E$ then $\mathscr{x}_{E \cup F}(B) $ increases by 1, $\mathscr{x}_{E}(B)$ increases by 1 also, the same can be said for $x \in F$ and $\mathscr{x}_{E \cup F}(B) , \mathscr{x}_{F}(B)$
in the case that $x \in E \cap F$ then $\mathscr{x}_{E \cup F}(B)$ remains the same and $\mathscr{x}_{E}(B) + \mathscr{x}_{F}(B) - \mathscr{x}_{E \cap F}(B)$ remains the same, since all double counted $b \in B$ are taken away.
For the cases not mentioned all functions do not change as they equal $0$


$\mathscr{x}_{E \cup F \cup G} = \mathscr{x}_{E} + \mathscr{x}_{F} +  \mathscr{x}_{G} - \mathscr{x}_{E \cap F} - \mathscr{x}_{E \cap G} - \mathscr{x}_{F \cap G}$ in this way all double counts are taken away and triple counts are also dealt with.

8. Let $A$ be a set to which there belongs precisely $n$ distinct objects. Prove that there are precisely $2^n$ distinct objects that belong to $2^{A}$.

$\lvert{A}\rvert = n, A = \{a_{1}, a_{2}, ..., a_{n} \}$, for all subsets $W \subset A, \lvert{W}\rvert = w$, 
where $w$ is the number representing the elements in $W$
$\forall a \in W$ either $a \in W$ or $a \not \in W$, so that to get the size of all combinations of subsets where order doesn't matter is $2^{\lvert{W}\rvert}$
$\lvert{W}\rvert =w,\lvert{2^{W}}\rvert=  2^{\lvert{W}\rvert}$.
$\therefore \lvert{2^{A}}\rvert = 2^{n}$ 
