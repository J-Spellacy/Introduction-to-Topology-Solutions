1. Let $f:A \rightarrow B$ be given. Prove the following:
(a) For each subset $X \subset A, X \subset f^{-1}(f(X))$.

Firstly, from the definition of a function: $f:A \rightarrow B, x \in A, f(x) \in B \iff x' = x \implies f(x) = f(x')$  essentially given a value $x$ there is only one unique $f(x)$ but it is not necessarily true that $x$ is unique for $f(x)$ $f(x) \in B, f^{-1}(f(x)) = X, X \subset A, x \in X$

given then that $X \subset A, f(X) =  y \iff \forall x \in X, f(x) = y, x \in f^{-1}(y)$ and that $\forall x, x' \in X, f(x) = f(x')$
$\therefore \forall x \in X, x \in f^{-1}(f(x))$
$\therefore X \subset f^{-1}(f(X))$

(b) For each subset $Y \subset B, Y \supset f(f^{-1}(Y))$.

given $Y \subset B$ and from the earlier definition we know that $f^{-1}(Y) = \{ x \mid x \in A, f(x) \in Y, x \in f^{-1}(f(x))\}$
$\therefore y \in f(f^{-1}(Y)) \iff y = f(x), x \in f^{-1}(Y)$
$\therefore y \in f(f^{-1}(Y)),  f(x) \in Y $
$\therefore y \in Y, f(f^{-1}(Y)) \subset Y$

(c) If $f:A \rightarrow B$ is one-one, then for each subset $X \subset A$, $$f^{-1}(f(X)) = X.$$
from part a, $X \subset A \therefore X \subset f^{-1}(f(X))$, given that $f(x)$ is one-one, $f(x) = y, f^{-1}(y) = x, \forall x, x' \in A, f(x) = f(x'), \forall y, y' \in B, f(y) = f(y')$
essentially imaging $f^{-1}(y) = x$ as a function which takes $y$ and maps to $x$ so behaves the same as the function in part a (only possible since $f:A \rightarrow B$ is one-one).
$\therefore \forall y \in f(X), f^{-1}(y) \in X$
$\therefore X \subset f^{-1}(f(X)) \subset X$
$\therefore f^{-1}(f(X)) = X$

(d) If $f: A \rightarrow B$ is onto, then for each subset $Y \subset B$, $$f(f^{-1}(Y)) = Y.$$
since $f: A \rightarrow B$ is onto, $f(A) = B, \forall y \in  f(A), y \in B$, $\forall y \in B, y \in f(A)$
using the proof in part b we have shown that $\forall y \in f(f^{-1}(Y)), y \in Y$ 
now $\forall y \in Y, y \in f(f^{-1}(Y))$ as per the definition of an onto or surjective function

2. Let $A = \{ a_{1}, a_{2}\}$ and $B = \{ b_{1}, b_{2}\}$ be two sets, each having precisely two distinct elements. Let $f:A \rightarrow B$ be the constant function such that $f(a) = b_{1}$ for each $a \in A$.
(a) Prove that $f^{-1}(f(a_{1})) \not = \{a_{1}\}$. [Thus it is usually the case that $f^{-1}(f(X))$ and $X$ are not equal.]

$f^{-1}(f(a_{1})) = \{ x \mid f(x) = f(a_{1})\} \therefore a_{2} \in f^{-1}(f(a_{1}))$
$a_{2} \not \in \{ a_{1}\} \therefore f^{-1}(f(a_{1})) \not = \{a_{1}\}$

(b) Prove that $f(f^{-1}(B)) \not = B$. [Thus it is usually the case that $f(f^{-1}(B))$ and $B$ are not equal.]

$B = \{ b_{1}, b_{2}\}, f^{-1}(B) = f^{-1}(\{ b_{1}, b_{2}\}) $ the function is always equal  to $b_{1} \therefore f^{-1}(B) = f^{-1}(b_{1})$
and since $\forall a \in f^{-1}(b_{1}), f(a) = b_{1}$ by definition of the inverse of a function and constant functions,
$\therefore f(f^{-1}(B)) = b_{1} \not = B$ only the  case when $B = \{ b_{1}\}$ (confusing if this is true based on the axiom of regularity, I.e. $x \not = \{ x \}$

(c) Prove that $f(\{a_{1}\} \cap \{ a_{2}\} ) \not = f(\{a_{1}\}) \cap f(\{ a_{2}\})$. [Thus it is not usually the case that $f(X \cap X') \not = f(X) \cap f(X')$.]

$\{ a_{1}\} \cap \{a_{2}\} = \emptyset$, $f(\{ a_{1}\} ) = b_{1}, f(\{a_{2}\}) = b_{1} \therefore f(\{a_{1}\}) \cap f(\{ a_{2}\}) = \{b_{1}\} \not =  \emptyset$
$\therefore f(X \cap X') = f(X) \cap f(X') \iff \forall x, x' \in X \cap X', f(x) = f(x')$

4. Let $f: A \rightarrow B$ be given and let $\{ Y_{\alpha}\}_{\alpha \in I}$ be an indexed family of subsets of $B$. Prove:
(a) $f^{-1}(\cup_{\alpha \in I}Y_{\alpha}) = \cup_{\alpha \in I} f^{-1}(Y_{\alpha}) $.

$f^{-1}(\cup_{\alpha \in I} Y_{\alpha}) = \{a \mid a \in A, \exists\alpha \in I\ s.t.\  f(a) \in Y_{\alpha} \}$ 
$a \in \cup_{\alpha \in I} f^{-1}(Y_{\alpha}) \iff a \in A, \exists\alpha \in I\ s.t.\  f(a) \in Y_{\alpha} $
since they hold equivelant conditions for truth, they only exist if they are equal
$\therefore f^{-1}(\cup_{\alpha \in I}Y_{\alpha}) = \cup_{\alpha \in I} f^{-1}(Y_{\alpha})$

(b) $f^{-1}(\cap_{\alpha \in I}Y_{\alpha}) = \cap_{\alpha \in I} f^{-1}(Y_{\alpha}) $.


$f^{-1}(\cap_{\alpha \in I} Y_{\alpha}) = \{a \mid a \in A, \forall\alpha \in I, f(a) \in Y_{\alpha} \}$ 
$a \in \cap_{\alpha \in I} f^{-1}(Y_{\alpha}) \iff a \in A, \forall\alpha \in I, f(a) \in Y_{\alpha} $
since they hold equivelant conditions for truth, they only exist if they are equal
$\therefore f^{-1}(\cap_{\alpha \in I}Y_{\alpha}) = \cap_{\alpha \in I} f^{-1}(Y_{\alpha})$

(c) If $X \subset B$ then $f^{-1}(\complement(X)) = \complement(f^{-1}(X))$.


$f^{-1}(\complement(X)) = \{ a \mid f(a) \in B, f(a) \not \in X\}$
$a \in \complement(f^{-1}(X)) \iff f(a) \not \in X, a \in A\ s.t.\ f:A \rightarrow B$ since $X \subset B, f^{-1}(X) \subset f^{-1}(B)$
$\therefore a \in \complement(f^{-1}(X)) \iff f(a) \not \in X, f(a) \in B$
since they hold equivelant conditions for truth, they only exist if they are equal
$\therefore f^{-1}(\complement(X)) = \complement(f^{-1}(X))$

(d) If $X$ is a subset of $A$, and $Y$ is a subset of $B$, then $f(X \cap f^{-1}(Y)) = f(X) \cap Y$.


